Veronderstel dat we het nulpunt willen vinden van een lineaire functie in \'e\'en variabele, f(x) = ax+b. We weten dat het nulpunt gelijk is aan $\frac{-b}{a}$. Hoe zou NR dit nu vinden? stel a=3 en b=2, en we starten met startwaarde $x_0$=4. We hebben f(4)=14 en  f'(4)=3. Dit wil zeggen dat als we X verhogen met 1, we f(x) verhogen met 3 (interpretatie afgeleide). Om vanaf de beginwaarde $x_0$ tot aan nul te raken, moeten we een vermindering van 14 hebben, daarvoor moet de volgende benadering $\frac{14}{3}$ lager liggen dan de huidige benadering. We passen nu NR toe \\
$x_1$ = $x_0$ - $\frac{f(x_0}{f'(x_0)}$ \\ 
dit geeft 4 - $\frac{14}{3}$ = $\frac{-2}{3}$ voor onze volgende benadering, wat gelijk is aan het nulpunt. Voor lineaire functies is het dus duidelijk dat deze benadering altijd zal werken in \'e\'en stap.
