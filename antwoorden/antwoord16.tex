De differentiefout wordt gegeven door de formule:
\[
D_n(x)= \frac{\pi'(x_i)}{(n+1)!} f^{(n+1)}(\xi(x_i))
\]
De wordt maximaal als $\pi'(x_i)$ maximaal wordt. $\pi(x)$ wordt gegeven door:
\[
\pi(x) = (x-x_0)(x-x_1)...(x-x_n)
\]
\[
\begin{array}{l}
\pi'(x) = 1(x-x_1)...(x-x_n)+(x-x_0)[(x-x_2)...(x-x_)  \\
+(x-x_1)[(x-x_3)...(x-x_n)+(x-x_2)[...]] \\
\pi'(x) =(x-x_1)...(x-x_n)+(x-x_0)(x-x_2)...(x-x_n)+  \\
(x-x_0)(x-x_1)(x-x_3)...(x-x_n)+...+ \\
(x-x_0)(x-x_1)...(x-x_{n-2})(x-x_{n-1})\\
\end{array}
\]
Deze formule is maximaal indien de factoren maximaal zijn. Dit is het geval als de punten op gelijke afstand van elkaar gelegen zijn.\\
\\
Dan geldt:
\[
D_n(x_i) = (-1)^{n-i}\frac{i!(n-i)!}{(n+1)!}h^{n}f^{(n+1)}(\xi(x_i))
\]