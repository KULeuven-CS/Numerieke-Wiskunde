Een nulpunt ($x^*$) is slecht geconditioneerd als $|f'(x^*)|$ klein is, en goed geconditioneerd anders.\\
$f(x) = x^2 + 2x + c$\\
Nulpunt: $x^* = \frac{-2 \pm \sqrt{4-4c}}{2}$\\
$f'(x) = 2x + 2$\\
Nulpunt invullen in afgeleide:\\
$f'(x^*) = 2(\frac{-2 \pm \sqrt{4-4c}}{2}) + 2$\\
$=> -2 \pm \sqrt{4-4c} + 2 = \pm \sqrt{4-4c}$ voor $c < 1$\\
Als c $\approx$ 1 is $\sqrt{4-4c}$ bijna 0, en is de wortel slecht geconditioneerd. (Absolute fout)\\
Voor de relatieve fout kunnen we de conditie hier niet nagaan, want we moeten delen door $f(x^*) = 0$.